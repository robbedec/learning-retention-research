%==============================================================================
% Sjabloon onderzoeksvoorstel bachelorproef
%==============================================================================
% Gebaseerd op LaTeX-sjabloon ‘Stylish Article’ (zie voorstel.cls)
% Auteur: Jens Buysse, Bert Van Vreckem
%
% Compileren in TeXstudio:
%
% - Zorg dat Biber de bibliografie compileert (en niet Biblatex)
%   Options > Configure > Build > Default Bibliography Tool: "txs:///biber"
% - F5 om te compileren en het resultaat te bekijken.
% - Als de bibliografie niet zichtbaar is, probeer dan F5 - F8 - F5
%   Met F8 compileer je de bibliografie apart.
%
% Als je JabRef gebruikt voor het bijhouden van de bibliografie, zorg dan
% dat je in ``biblatex''-modus opslaat: File > Switch to BibLaTeX mode.

\documentclass{voorstel}

%------------------------------------------------------------------------------
% Metadata over het voorstel
%------------------------------------------------------------------------------

%---------- Titel & auteur ----------------------------------------------------

% TODO: geef werktitel van je eigen voorstel op
\PaperTitle{Doolhofonderzoek bij selectief herinneren}
\PaperType{Onderzoeksvoorstel Onderzoekstechnieken 2018-2019} % Type document

% TODO: vul je eigen naam in als auteur, geef ook je emailadres mee!
\Authors {Robbe Decorte\textsuperscript{1}, Sander Baele\textsuperscript{2}, Niek Gasthuys\textsuperscript{3}, Jurgen De Groote\textsuperscript{4}} % Authors
\CoPromotor{}
\affiliation{\textbf{Contact:}
	\textsuperscript{1} \href{mailto:robbe.decorte@student.hogent.be}{robbe.decorte@student.hogent.be};
	\textsuperscript{2} \href{mailto:sander.baele@student.hogent.be}{sander.baele@student.hogent.be};
	\textsuperscript{3} \href{mailto:niek.gasthuys@student.hogent.be}{niek.gasthuys@student.hogent.be};
	\textsuperscript{4} \href{mailto:jurgen.degroote@student.hogent.be}{jurgen.degroote@student.hogent.be};
}

%---------- Abstract ----------------------------------------------------------

\Abstract{Door deze testen uit te voeren zijn we in staat om 2 individuele herinneringmethoden die reeds bevestigd zijn aan elkaar te linken. Is het mogelijk om je geheugen te trainen om een mentaal pad te volgen en zo de woorden die je onderweg tegenkomt kan herinneren. We onderzoeken verschillende testpersonen aan de hand van een doolhof op papier om zo een pad te volgen \autocite{CohenSquire1980} onderweg liggen kaartjes met een woord op. Bij het begin van de sessie herhalen we alle woorden die niet gepasseerd zijn op het bord \autocite{BuschkeFuld1974}. Volgens onze hypothese zou het pad dat je volgt moeten helpen om de volgorde van de woorden te herinneren. Mocht deze hypothese kloppen dan bevestigen we de 2 theoriën die hier zijn toegepast en kan je dus het geheugen trainen om een link te leggen tussen verschillende soorten informatie.
}

%---------- Onderzoeksdomein en sleutelwoorden --------------------------------
% TODO: Sleutelwoorden:
%
% Het eerste sleutelwoord beschrijft het onderzoeksdomein. Je kan kiezen uit
% deze lijst:
%
% - Mobiele applicatieontwikkeling
% - Webapplicatieontwikkeling
% - Applicatieontwikkeling (andere)
% - Systeembeheer
% - Netwerkbeheer
% - Mainframe
% - E-business
% - Databanken en big data
% - Machineleertechnieken en kunstmatige intelligentie
% - Andere (specifieer)
%
% De andere sleutelwoorden zijn vrij te kiezen

\Keywords{Onderzoeksdomein. Onderzoekstechnieken --- Learning --- Retention} % Keywords
\newcommand{\keywordname}{Sleutelwoorden} % Defines the keywords heading name

%---------- Titel, inhoud -----------------------------------------------------

\begin{document}
	
	\flushbottom % Makes all text pages the same height
	\maketitle % Print the title and abstract box
	\tableofcontents % Print the contents section
	\thispagestyle{empty} % Removes page numbering from the first page
	
	%------------------------------------------------------------------------------
	% Hoofdtekst
	%------------------------------------------------------------------------------
	
	% De hoofdtekst van het voorstel zit in een apart bestand, zodat het makkelijk
	% kan opgenomen worden in de bijlagen van de bachelorproef zelf.
	%---------- Inleiding ---------------------------------------------------------
	
	\section{Introductie} % The \section*{} command stops section numbering
	\label{sec:introductie}

	Het achteruit gaan van het geheugen is een fenomeen waar vroeg of laat iedereen last van heeft, de mate is deels te wijten aan beslissingen van de persoon zelf maar dit kan plots gebeuren door ziekte, ongeval ... In dit voorstel gaan we op zoek naar data die kan bevestigen dat je door middel van gedachten te linken je meer kan onthouden zonder rekening te houden met problemen aan het geheugen. Het geheugen is een zeer actueel onderwerp en daarom kan deze studie makkelijk gelinkt worden aan onderzoekstechnieken die we overnemen van studies die zich reeds hebben bewezen.
	
	%---------- Stand van zaken ---------------------------------------------------
	
	\section{Literatuurstudie}
	%\label{sec:literatuurstudie}
	\subsection{Robbe Decorte - Literatuur}
	Evaluating storage, retention, and retrieval in disordered memory and learning \autocite{BuschkeFuld1974}
	
	Ze bespreken 2 simpele manieren om bij verschillende mensen het geheugen te quoteren en met
	elkaar te vergelijken (nl. selectief herinneren en restrictief herinneren)
	Bij selectief herinneren herhalen ze enkel de woorden die je niet genoemd hebt in de test die direct volgt na het
	opzeggen van alle woorden tot je 2 opeenvolgende testen alle woorden kan opzeggen, restrictief gaat doortot je elk woord minstens
	1x kan herinneren zodat de lijst inkrimpt (elk woord wordt na het herinneren ervan uit de lijst gehaald).
	Woorden die je kan herinneren in een sessie waar die voorgaand niet in gezegd is, komen uit je lange
	termijn geheugen. Het falen van de patiënt kunnen we linken aan het niet (snel genoeg) kunnen onthouden van woorden in het
	lange termijn geheugen. We kunnen zeggen dat een lijst gekend is als de patiënt consistent alle woorden kan herhalen, het heeft dus geen nut als
	je uit het lange termijn geheugen woorden kan onthouden wanneer je andere woorden die net gezegd zijn niet kan herhalen.
	Deze methoden zijn enkel nuttig om abnormaliteiten vast te stellen (in dit geval alcoholisme) maar vertellen weinig over de staat van het geheugen
	als je 2 patiënten met hetzelfde probleem met elkaar vergelijkt. De patiënt had bij beide methoden 6 pogingen nodig vooraleer die in staat was om alle woorden
	op te zeggen, maar was niet in staat om deze te herhalen in de volgende sessie van selectief herinneren. De beste manier om dit aan te pakken is door je visie te 
	veranderen, je mag het niet zien als 10 individuele woorden maar als een lijst waar je zelf een link moet leggen tussen verschillende woorden. Op deze manier kunnen wij
	een hypothese opstellen, "Door een link te leggen tussen de woorden zorg je niet alleen dat je in de volgende sessie een hoger aantal woorden kan herhalen maar ook 
	dat je na enkele uren meer woorden kan opsommen dan iemand die deze manier niet toepast". Deze hypothese lijkt me belangrijk voor het resultaat maar is niet behandeld in deze studie.
	

	Preserved learning and retention of pattern-analyzing skill in amnesia: dissociation of knowing how an knowing that \autocite{CohenSquire1980}
	
	Er wordt onderzocht in hoeverre een persoon verschillende soorten informatie kan onthouden terwijl hij aan geheugenverlies lijdt, de 2 soorten informatie die ze hier onderscheiden zijn
	handelingen waarvan je weet hoe ze werken (data based) en informatie waarvan je weet dat het iets doet, zoals bijvoorbeeld de naam van een persoon (hier is geen logica aan verbonden).
	De hypothese stelt dat geheugenverlies minder effect heeft op de 1e soort dan initeel gedacht. Ze onderzoeken dit door mensen met een sterke vorm van geheugenverlies enkele zaken laten
	uitvoeren, dit telkens iets fysiek (nl. traceren en spiegelen). Je ziet in de resultaten dat dit elke poging verbeterde terwijl de patiënt zich niet / amper kon herinneren dat hij deze 
	handelingen eerder had uitgevoerd. In een 2e experiment laten ze de patiënten 3 maanden lang met 3 sessies per dag lezen, door deze skill te trainen verliezen ze niet het vermogen om te
	lezen terwijl dit bij een normaal geval wel het geval is. Deze mensen kunnen hun geheugen dus trainen op motor skills zonder dat ze deze onthouden, ze linken dit aan een invloed van het
	zenuwstelsel waar enkele simpele handelingen een soort van genaturaliseerd worden. De data die hier is gevonden geeft ons niet enkel een beter beeld van de ziekte zelf maar zou moeten
	helpen bij het opzetten van AI en het bekijken van cognitieve wetenschappen binnen de computerwereld, er zijn namelijk al jaren discussies over hoe ze dat fenomeen moeten beschrijven.
	
	\subsection{Sander Baele - Literatuur}
	Team-based learning enhances long-term retention and critical thinking in an undergraduate microbial physiology course \autocite{Mcinerney2003}
	
	Team-based learning vereist een speciale aanpak. Het vereist dat het vak geherstructureerd word om nieuw gevormde groepen en teams uit de dagen met complexe, uitdagende, leerrijke taken. Er zijn algemeen gezien 3 aanpakken die zich goed van elkaar onderscheiden, namelijk: coöperatief leren, probleem-gebaseerd leren, en team-gebaseerd leren.
	Coöperatief leren vereist weinig herstructurering aan het opleidingsonderdeel. Er worden groepen gevormd om een bestaand onderdeel uit het opleidingsonderdeel te volbrengen.
	Probleem-gebaseerd leren vereist een grotere herstructurering van het opleidingsonderdeel. Het probleem komt er, nog voordat alle relevante materie gezien is. Deze aanpak bevindt zich tussen coöperatief leren en team-gebaseerd leren.
	Team-gebaseerd leren is op zich gelijkaarde aan probleem-gebaseerd leren, maar studenten vergaren eerst de nodige leerstof. Dit gebeurd voor men begint aan het probleem. Er worden geen rollen verdeeld, en tussenkomst van een leerkracht is optioneel.
	De eerste fase van de 3, is de opzet fase. Tijdens deze fase vergaren de studenten de introductie informatie van lesmomenten, lezingen, enz. en worden ze getest op deze informatie. Men wordt individueel getest. Erna maken ze direct dezelfde test in groep. Beide testen worden gescoord en tellen mee voor de uiteindelijke scores. Hierop volgt nog feedback van de leerkracht indien nodig.
	De volgende fase is de toepas fase. De teams krijgen toenemend uitdagende projecten, dit zijn louter oefeningen, en tellen niet mee voor de score voor het vak. De leerkracht gaat met de studenten in gesprek over hun conclusies, en vergaart zo feedback over de kwaliteit van de antwoorden. 
	De derde en laatste fase is de scorings fase. Deze fase zal bijdragen van 20 tot 40 procent van de score die de student krijgt voor het vak. Uiteraard volgt er ook een peer evaluation, waarin de studenten elkaar kunnen beoordelen op basis van elkanders bijdrage.
	De kwaliteit van de projecten was het eerste criterium die belangrijk was om de educatieve waarde te bepalen van dat project. Deze waarde bepalen omvatte meerdere pijlers, zoals het detail van analyse, de hoeveelheid materie uit het opleidingsonderdeel, de integratie van materie uit andere vakken, en creatieve en intuïtieve deducties die teams zelf maakten.
	De finale testen werden zodanig opgesteld dat ze qua formaat, uitgebreidheid, en moeilijkheidsgraad, gelijk waren tussen jaren mét en jaren zonder groeps-gebaseerd leren.
	Er werden 2 vragen gesteld in jaren mét en jaren zonder groeps-gebaseerd leren. Namelijk de hoeveelheid informatie de student zelf vond dat hij/ zij opgenomen had, en de score die hij/ zij gaf aan de leerkracht met betrekking tot de mate waarin die leerkracht kritisch denken promootte.
	Er werd Chi-square analyse gebruikt om de significantie hiervan te bepalen; die was 0.05.
	Men begon het eerste jaar (2000) met toevoeging van quizzen. Deze zorgden voor een positieverer attitude, maar qua scores presteerden de studenten nog vrij laag. Het was pas dankzij de team-gebaseerde projecten dat deze omhoog gingen.
	In het jaar 2001 en 2001 gebruikte men de projecten. Men zag dat evenveel studenten boven de 90 procent zaten als de jaren voordien. Aan de hand van Chi-square analyse kon men dus aantonen dat de moeilijkheidsgraad dezelfde was gebleven. Het aantal studenten die tussen 70 procent en 90 procent score, steeg wel aanzienlijk. Meer dan 80 procent van de studenten was positief ten opzichte van het opleidingsonderdeel, de docent, en de team activiteiten.
	Men kan wel degelijk vaststellen dat team-gebaseerd leren voordelen heeft bij het verwerken van grote hoeveelheden informatie. We moeten wel meegeven dat dit zo blijkt bij een opleidingsonderdeel microbiologie. Dit kan natuurlijk anders zijn bij andere soorten materie.
	
	Memory vocabulary learning strategies and long-term retention \autocite{Nemati2009}
	
	Volgens een studie van \autocite{Nielson (2006)}, is het beter om in de vroegere stadia van het aanleren van een taal, 'de-contextualized' woorden aan te leren. Hij toont aan dat dit beter is om een stevige basis te leggen. Naarmate men vordert in de taal, zou men gelijdelijk aan over moeten gaan naar 'contextualized' leren. Een hypothese van \autocite{Craik and Tulving (1975)} suggereert dat het niet belangrijk is hoe recent woorden geleerd zijn, maar hoe diep men het woord in zich neemt. Dit betekent, hoe meer de persoon in kwestie het woord manipuleert, en cognitief opneemt, hoe beter het behoud ervan. In deze studie zijn de geheugen strategieën onze grootste focus, opgedeeld in 4 sets. Deze sets zijn ``aanmaken van mentale linken'', ``toepassen van beeld en geluid'', ``goed hernemen'', en ``gebruik maken van acties''. Deze klassen werden opgedeeld door \autocite{Oxford (1990)}. We gaan ons nog meer bezighouden met drie substrategieën: groeperen, acroniemen maken, en beelvorming. Het groeperen verwijst naar het klasseren van taal in betekenisvolle contexten, om het zo makkelijker te maken om te herinneren. Acroniemen gebruiken is volgens Oxford nieuwe woorden in een context zetten, om ze beter te herinneren. Als laatste kunnen we ook beeldvorming gebruiken om opnemen van nieuwe woorden te faciliteren.
	Enkele onderzoeksvragen die we gesteld hebben zijn de volgende:
	Is er een verschil tussen de prestaties van de leerling in elke sub-strategie en de strategie die de leerling volgens zichzelf gebruikt?
	Is er een impact van aanleren van sub-strategies aan leerlingen ten opzichte van de control groups?
	De proefpersonen van deze studie zijn 310 vrouwelijke studenten, die nog niet aan hoger onderwijs doen. Het onderzoek werd in India uitgevoerd. De moedertaal daar, is Kannada. Deze taal werd gebruikt om de studenten Engels aan te leren. De studenten waren tussen 16 en 18 jaar. Uiteindelijk ware ner 140 studenten in de control group, en 170 in de experimentele groep.
	Om data te verzamelen werd aan de studenten gevraagd een vragenlijst in te vullen en aan te geven hoeveel ze sommige strategieën denken gebruikt te hebben.
	Er werd ook gebruikt gemaakt van een Vocabulary Knowledge Scale (VKS). Dit om een idee te krijgen van de vooruitgang in kennis.
	De studie verliep in 3 stappen: De eerste stap is de pre-test. 
	Dit is de eerder vermelde zelf-evaluatie vragenlijst en de VKS. 
	Bij deze pre-test horen ook nog de lessen.
	Lessen in de control group gebruikte verschillende methoden, zoals de presentatie van individuele woorden, orale uitspraak, de woorden op het bord schrijven, een kleine beschrijving ervan, synoniemen en antoniemen aanhalen, en minimale contexten (zoals een kleine zin met het woord in).
	In de experimentele groep begon men met de student kennis te laten maken met strategieën, en zo een strategy awareness te creëren. Bij de eerste strategie (groeperen), moesten de studenten woorden bij hetzelfde thema plaatsen. Bij de tweede strategie, moesten de studenten voor woorden een afkorting maken, om zo de woorden beter te onthouden. Bij de derde strategie, moesten de studenten het visueel voorstellen.
	De tweede stap van het onderzoek, is de post-test 1. Na een lesmoment wordt een korte termijn test gegeven.
	De derde stap van het onderozke, is de post-test 2. Na 2 weken werd er weer een test gegeven op de kennis.
	De eerste onderzoeksvraag was de vraag waarbij studenten zelf moesten evalueren hoezeer ze de strategieën gebruikten. Er bleek dat de control groups overschatten hoeveel ze de strategiën gebruikten. Uiteindelijk deden de experimentele groepen het iets beter dan de control groups. En na de twee weken gingen de scores van de experimentele groepen ook minder achteruit.
	Ook is het duidelijk dat er wel degelijk een impact is van het aanleren van de strategieën aan de leerlingen
	We concluderen dat het toepassen van deze strategieën het opnemen van nieuwe woorden beïnvloed op een positieve manier. Deze diepere manier van aanpakken, en aanleren op een efficiëntere manier, heeft positieve gevolgen voor het behoud van de opgedane kennis.
	
	\subsection{Niek Gasthuys - Literatuur}
	\subsection{Jurgen De Groote - Literatuur}
	
	%---------- Methodologie ------------------------------------------------------
	\section{Methodologie}
	\label{sec:methodologie}
	
	Om zo veel mogelijk data op hetzelfde moment te verzamelen werken we een simulatie/experiment in een klasgroep. Op het bord schetsen we een doolhof en plakken we op 10 verschillende plaatsen een kaartje met een woord op omgekeert aan het bord, de volgorde is van belang. Bij aanvang van de sessie draaien we alle kaarten om en krijgt ons testpersoon 5 minuten om het bord te bekijken zonder dat hij externe hulpmiddelen (zoals bijvoorbeel een blad papier en potlood) mag gebruiken. Als de tijd voorbij is draaien we alle kaarten terug om en mag de persoon beginnen. Hij zit voor het bord en volgt de weg van het doolhof en elke kaart die hij tegenkomt moet hij benoemen tot hij aan het einde geraak, om de tijdsdruk te verlagen zien we meer dan 5 pogingen als gefaald. Na elke sessie draaien we de kaarten om die niet of fout benoemd zijn en krijgt hij 15 seconden per woord om zijn huidige links verder op te bouwen. Deel 2 van het experiment is met dezelfde persoon maar het bord valt weg. Er is iemand die 10 nieuwe woorden 1 voor 1 voorleest met een pauze van 3 à 5 seconden tussen elk woord. Je krijgt hier terug 5 pogingen om de 10 woorden in juiste volgorde te herinneren, ook hier wordt gewerkt met selectief herinneren \autocite{BuschkeFuld1974}.
	
	%---------- Verwachte resultaten ----------------------------------------------
	\section{Verwachte resultaten}
	\label{sec:verwachte_resultaten}
	
	Hier beschrijf je welke resultaten je verwacht. Als je metingen en simulaties uitvoert, kan je hier al mock-ups maken van de grafieken samen met de verwachte conclusies. Benoem zeker al je assen en de stukken van de grafiek die je gaat gebruiken. Dit zorgt ervoor dat je concreet weet hoe je je data gaat moeten structureren.
	
	%---------- Verwachte conclusies ----------------------------------------------
	\section{Verwachte conclusies}
	\label{sec:verwachte_conclusies}
	
	Hier beschrijf je wat je verwacht uit je onderzoek, met de motivatie waarom. Het is \textbf{niet} erg indien uit je onderzoek andere resultaten en conclusies vloeien dan dat je hier beschrijft: het is dan juist interessant om te onderzoeken waarom jouw hypothesen niet overeenkomen met de resultaten.
	
	
	%------------------------------------------------------------------------------
	% Referentielijst
	%------------------------------------------------------------------------------
	% TODO: de gerefereerde werken moeten in BibTeX-bestand ``voorstel.bib''
	% voorkomen. Gebruik JabRef om je bibliografie bij te houden en vergeet niet
	% om compatibiliteit met Biber/BibLaTeX aan te zetten (File > Switch to
	% BibLaTeX mode)
	
	\phantomsection
	\printbibliography[heading=bibintoc]
	
\end{document}